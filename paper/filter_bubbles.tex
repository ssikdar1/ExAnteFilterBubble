\documentclass[format=acmsmall, review=false]{acmart}
\usepackage{acm-ec-20}
\usepackage{booktabs} % For formal tables
\usepackage[ruled]{algorithm2e} % For algorithms

\usepackage{subcaption}
\newcommand{\xhdr}[1]{\vspace{1mm} \noindent{\bf #1}}

\renewcommand{\algorithmcfname}{ALGORITHM}
\SetAlFnt{\small}
\SetAlCapFnt{\small}
\SetAlCapNameFnt{\small}
\SetAlCapHSkip{0pt}
\IncMargin{-\parindent}

% Choose a citation style by commenting/uncommenting the appropriate line:
%\setcitestyle{authoryear}
\setcitestyle{acmnumeric}

\newtheorem{finding}{Finding}

% Title. Note the optional short title for running heads. In the interest of anonymization, please do not include any acknowledgements.
\newcommand\PaperTitle{Deconstructing the Filter Bubble: \\ Consumer Decision-Making and Recommender Systems}
\title[\PaperTitle]{\PaperTitle}

% Anonymized submission.
\author{Submission 20}

% Abstract. Note that this must come before \maketitle.
\begin{abstract}
We study a model of user decision-making in the context of recommender systems via numerical simulation. We explore the extent to which two issues traditionally associated with recommender systems -- \textit{filter bubbles} and \textit{user homogenization} -- can be attributed to recommendation systems. In our model users face a sequential decision problem under uncertainty and there are informational spillovers where a user consuming a product and learning how she values it reduces uncertainty and provides information about similar products. We find that this generates behavior consistent with filter bubble effects -- where users consume increasingly similar items over time -- and that recommendation can help alleviate these effects, as has been found in the empirical work of Nguyen et al. (2014). However, we find that recommendation, while alleviating filter bubble effects, leads to an increase in user homogeneity. Our results highlight the path-dependence of user choice in environments where recommender systems are deployed.
\end{abstract}

\begin{document}

\maketitle


% Paper body
\section{Introduction}

Recommender Systems (RS) have become critical for assisting users in navigating the large choice sets that they face in many online markets. For instance, users have to select from thousands of movies on Netflix, millions of products on Amazon, and billions of videos on YouTube. Users in many cases are not aware of most items, let alone have full information about their preferences over them and, to make matters worse, the goods in these markets are usually experience goods whose true value can only be learned after consumption. Furthermore, users interact with these systems routinely and watch more than one movie on Netflix, buy more than one product on Amazon, and listen to more than one video on YouTube.
\par

Recommender systems have been influential in shaping consumer choice in these markets with 75\% of movies watched on Netflix and 35\% of page-views on Amazon coming from recommendations.\footnote{MacKenzie et al. (2013, Oct.),  How retailers can keep up with consumers. \url{https://www.mckinsey.com/industries/retail/our-insights/how-retailers-can-keep-up-with-consumers}. Retrieved on October 3, 2019.} While there are many positive effects from these systems, there is an increasing worry that there are unintended side-effects of recommendation systems. There have been claims that YouTube's recommendation algorithm unintentionally lead to the radicalization of many individuals,\footnote{Friedersdorf (2018, Mar. 12), YouTube Extremism and the Long Tail, \textit{The Atlantic}. \url{https://www.theatlantic.com/politics/archive/2018/03/youtube-extremism-and-the-long-tail/555350}. Retrieved on October 3, 2019.} that personalized recommender systems lead users into \textit{filter bubbles} where they effectively get isolated from a diversity of viewpoints or content \cite{pariser2011filter}, and that personalized recommender systems may also lead users to become increasingly homogenized at the same time \cite{chaney2018algorithmic, hosanagar2013will}.
\par

\xhdr{Model} We consider a model of user choice that has the following components.
\par
The first component of our model is that users sequentially consume items and face large choice sets. In our setting of interest users are long-lived, but they only consume a small fraction of this choice set over their lifetime.
\par
The second component of our model is that users have \textit{uncertainty} about their valuation of the different items. This is motivated both by the fact that many markets where recommender systems are deployed contain experience goods, whose true value can only be learned after consumption, and the fact that such uncertainty is why recommender systems exist in the first place. Thus, users face a sequential decision-making problem under uncertainty.
\par 
The third, and key, component of our model is that consumption of an item reveals information that changes user beliefs about the valuation of similar items. Unlike in standard sequential decision-making problems, in our model once an item is consumed all uncertainty about its valuation is resolved. This valuation provides information for consumers that enables them to update their beliefs about similar items. This exploits the fact that the valuations of similar items are correlated which assists users in navigating the vast product space. The idea that users make similarity-based assessments to guide their choice has grounding in theoretical work in decision theory \cite{gilboa1995case} and empirical evidence of how users navigate large choice sets \cite{schulz2019structured}.
\par 
The fourth component of our model is that recommendation provides users with information about the true valuations. We model the realized valuations as being a weighted sum of a common-value and an idiosyncratic component. This formulation gives a stylized notion of predictability of user preferences where the idiosyncratic component is inherently unpredictable given other individuals' preferences and the common-value component is what the recommender can learn from previous users' data. We suppose that the recommender knows the common-value component for each item and combines it with individuals' beliefs over the product space when designing personalized recommendation.

\xhdr{Contributions}
Using this model we first investigate the degree to which recommender systems lead users into filter bubbles. Our model rationalizes the empirical results in \cite{nguyen2014exploring} who show that in the context of movie consumption, user behavior is consistent with filter bubble effects \textit{even without recommendation}. The driving force of this in our model is that preferences for similar goods are correlated, which implies that when a good is consumed and the user learns its value, it provides information about similar goods. Crucially, this not only impacts the underlying belief about the average utility of the good, but also the level of uncertainty. Consequently, this learning spillover induces users to consume goods ``similar'' to those they consumed before that had high realized value, leading to an increasing narrowing of consumption patterns towards these regions of the product space. This effect is further amplified when users are \textit{risk-averse}, a concept from decision theory where all else being equal, users have a preference for goods with lower uncertainty to those with higher uncertainty.
We compare this benchmark to the case where recommendations are provided. 
We consider recommendations as giving information to users in order to reduce their uncertainty.
Recommendations could potentially make matters worse: providing information about the value of the goods when this value is correlated could intensify local search and decrease consumption diversity. However, and in line with the empirical results from \cite{nguyen2014exploring}, we find that recommendations reduce the narrowing effect.
\par

We then explore how diverse the overall set of items that users consume is. It could be expected that higher diversity being associated with more intense search is associated with higher welfare. Much to the opposite, we find that, without recommendation, higher diversity of items does not ensure higher welfare.
This is due to the fact that once users find good items, they consume items nearby in the product space due to the correlation of values, leading to higher overall value from consumption. When they consume products they value little, they move to another part of the product space. A sequence of bad realizations will then lead to very intense search over very different parts of the product space and high consumption diversity, whereas discovering a sweet spot leads to highly rewarding local search and very low diversity. Thus, contrary to what a natural intuition could suggest, we find that consumption set diversity is not by itself a useful measure for understanding the impact of recommender systems on users. 
\par

We find that recommendation can lead to increases in homogeneity among users. We model the realized valuations as being a weighted sum of a common-value and an idiosyncratic component. This formulation gives a stylized notion of predictability of user preferences where the idiosyncratic component is inherently unpredictable given other individuals' preferences and the common-value component is what the recommonder can learn from previous users' data. The recommender uses its knowledge of the common-value component and the individuals' beliefs over the product space when designing personalized recommendation. This leads to an increase in welfare for users, but concentrates their consumption around items that have a high common-value component, resulting in an increase in homogeneity.
\par

Finally, we discuss how our model and findings can be used to inform the design and evaluation of recommender systems. Our evaluation metric relies on the intuition that a recommendation is only useful insofar as it leads a user to take a better action than she would without the recommendation. Thus, in order to evaluate the value of recommendation, it is important to be able to predict what a user would do without recommendation and how recommendation would change their behavior. If recommendations fail to affect behavior, they are of no use: the user would consume the same good and learn its value with and without the recommendation. Moreover, we argue that successful recommendation systems have to take into account the degree of uncertainty users have over different portions of the product space. Being able to predict valuations after consumption is not sufficient as uncertainty over the idiosyncratic component of one's own preferences may steer the user away from the recommended items in the face of risk-aversion.
\par

\textbf{Related Work.} 
\cite{pariser2011filter} first informally described the ``filter bubble'' which is the idea that online personalization services would lead users down paths of increasingly narrower content so that they would effectively be isolated from a diversity of viewpoints or content. A number of studies, done in a wide range of disciplines, have since empirically studied whether or not this phenomenon exists in various contexts \cite{flaxman2016filter,hosanagar2013will,moller2018blame,nguyen2014exploring}. The most relevant to our study is \cite{nguyen2014exploring} who study whether this effect exists with recommender systems in the context of movie consumption. They find that even users whose consumption choices are not guided by recommendations fall into behavior consistent with ``filter bubbles'' and that recommender systems can actually increase the diversity of the content that users consume. To our knowledge there are no theoretical models that rationalize the empirical findings but predominantly empirical studies. Our paper provides a theoretical framework through which to view this problem.
The second question is to what extent recommender systems can lead users to become increasingly homogenized or consume similar content as each other. \cite{celam2008hits,treviranus2009value} show that incorporating content popularity into recommender systems can lead to increased user homogenization. \cite{chaney2018algorithmic} more generally look at the issue of user homogenization in recommender systems. Our results are similar to previous work, where the limits of recommendation lead the recommender to over-recommend products from certain portions of the product space which leads users to consume goods in those regions.
\cite{fleder2009blockbuster} also consider the effects of recommendation on multi-period user behavior, but predominantly focus on sales diversity and employ a different model of user behavior than we do here. \cite{hodgson2019horse} consider a similar model as ours where users engage in ``spatial learning'' and exploit the correlation of their preferences in the environment, but consider it in the context of search for a single item.
Our discussion of recommender system evaluation contributes with insights closely tied to the value of information that is used in the economics literature (see e.g. \cite{bergemann2019information}) and builds off the large literature stemming from \cite{mcnee2006being} that makes the case that accurate predictions of true realized valuations may not lead to the best recommendations. Our evaluation measure is most closely related to the notion of serendipity in recommender systems that has been studied extensively (e.g. see \cite{kotkov2016survey} for a survey).


\section{Our Model and Preliminaries}
\subsection{Preliminaries on Expected Utility Theory}
\noindent For every item $n$ in the product space $\mathcal J$, we assume that each user $i$ assigns a monetary equivalent $x_{i,n} \in \mathbb R$ to the experience of consuming it. Each user can value the same item differently. However, we assume that users have the same utility over money, given by a utility function $u: \mathbb R \to \mathbb R$, strictly increasing and continuous. So, ex-post, the value of item $n$ for user $i$ is given by $u(x_{i,n})$ However, before consuming the item, the user does not know exactly how she will value it. In particular, even users that will end up having the same ex-post valuation of item $n$ may differ in their ex-ante valuation because they hold different beliefs about it. We denote by $p_{i}$ the beliefs item $i$ has about how she will value each of the items in the product space. Note that this implies that consuming item $n$ is the same as taking a gamble. Each user evaluates the item according to the associated expected utility associated with the item, i.e. $U_i(n)=\mathbb E_{p_i}[u(x_n)]$. 
\par

Risk aversion captures the how different users react to the risk associated to a particular consumption opportunity. It is formalized as follows: a given gamble $x$ takes real values and follows distribution $p$. Then, for every gamble $x$, there is a certain amount of money that makes the user indifferent between taking the gamble or taking the sure amount of money. This sure amount of money is called the \textit{certainty equivalent} of gamble $x$ and is denoted as $\delta(x)$. A user $i$ is more risk-averse than another user $j$ when for every gamble $x$, user $i$ requires a lower value to give up the gamble when compared to user $j$, i.e. $\delta_i(x)$. Therefore, a more risk-averse user is more willing to avoid the risk of taking the gamble. We assume that the utility function takes a flexible functional form $u(x)=1-\exp(-\gamma x)$ for $\gamma\ne0$ and $u(x)=x$ for $\gamma=0$ -- known as constant absolute risk-aversion preferences (from hereon CARA). Higher $\gamma$ implies higher risk-aversion with $\gamma=0$ corresponding to the risk-neutral case and $\gamma>0$ to the risk-averse one. Our formulations here follow standard economic decision-theory (see \cite{mas1995microeconomic} for a textbook treatment of these topics).
\par

\subsection{Model}
\par
\xhdr{Users.} We consider a set of users $I$ where each user $i \in I$ faces the same finite set of $N$ items $\mathcal J = \left\{0,1,...,N-1\right\}$. For simplicity, we assume that users only derive pleasure from item $n \in \mathcal{J}$ the first time they consume it.
\par

We denote by $x_{i,n}$ user $i$'s realized value from consuming item $n$. In particular, we consider that the realized value derived from a given item can be decomposed in the following manner: $x_{i,n}= v_{i,n} + \beta v_n$, where $v_{i,n}$ denotes an idiosyncratic component -- i.e. user $i$'s idiosyncratic taste for product $n$ --  and $v_{n}$, a common-value component. One can interpret $v_n$ as a measure of how much item $n$ is valued in society in general and, in a sense, $v_{i,n}$ denotes how $i$ diverges from this overall ranking. The scalar $\beta \in \mathbb{R}_{t}$ denotes the degree to which valuations are idiosyncratic to each user or common across users. If $\beta=0$, it is impossible to generate meaningful predictions of any one's individual preferences based on others, while if $\beta$ is large, every individual has similar preferences.
\par
Stacking values in vector-form, we get the vector of values associated with each item 
$${\left(x_{i,n}\right)}_{n \in \mathcal{J}}=:X_i =V_i+ \beta V $$
where $V_i ={\left(v_{i,n}\right)}_{n \in \mathcal{J}}$ and $V={\left(v_{n}\right)}_{n \in \mathcal{J}}$.
\par
\xhdr{User Decision-Making.}
We assume the user makes $T$ choices and therefore can only consume up to $T$ items, where $T$ is a small fraction of $N$. This captures the idea that users are faced with an immense choice set but that ultimately they end up experiencing (and learning) about just a small fraction of it. Since this is a sequential decision-making problem under uncertainty, in principle users face an exploration-exploitation trade-off. However, for tractability, we impose that users are myopic and every period consume the product that they have not yet tried ($n_i^t$) that gives them the highest expected utility given the information from past consumption offers ($C_i^{t-1}=(n_i^1,...,n_i^{t-1})$) and their initial beliefs.
\par
\xhdr{User Beliefs.} We assume that all the true realized values are drawn at $t = 0$. However, users do not know the realized values before consuming an item, but rather have beliefs over the realized utilities.
Formally, user $i$ starts with some beliefs about $X_i$, namely that the idiosyncratic and common-value parts of the valuations are independent -- $V_i \perp \!\!\! \perp V$ -- and that each is multivariate normal 
\begin{enumerate}
\item $V_i \sim \mathcal N (\overline V_i, \Sigma_i)$ 
\item $V \sim \mathcal N(\overline V, \Sigma)$ with $\overline V =0$
\end{enumerate}
We impose the normality assumption for two reasons. The first is that this allows for simple and tractable belief updating. The second is that it allows us to incorporate an easily interpretable correlation structure between the items. The precise formulation of $\Sigma$ and $\Sigma_i$ that we consider is defined below when we discuss user learning.
\par
Keeping with the assumption that $V_i$ represents idiosyncratic deviations from $V$, we assume that, on the population level, prior beliefs $\overline V_i=\left(\overline v_{i,n}\right)_{n \in \mathcal{J}}$ are drawn independently from a jointly normal distribution, where $\overline v_{i,n} \sim \mathcal N (0, \overline \sigma^2)$ are independent and identically distributed. These $\overline v_{i,n}$ denote the prior belief that individual $i$ holds about her valuation over item $n$. As people are exposed to different backgrounds, their beliefs about how much they value a given item also varies and $\overline v_{i,n}$ denotes this idiosyncrasy at the level of prior beliefs.
\par

We assume users are expected utility maximizers. User $i$'s certainty equivalent for item $n$, the sure value that makes user $i$ indifferent between it and consuming the item $n$, conditional on the consumption history, is given by
$\delta_{i}(n)\mid C_i^{t-1}=\mu_n-\frac{1}{2}\gamma \Sigma_{nn}$, where $\mu_n$ and $\Sigma_{nn}$ are the expected value and variance for item $n$ that the user has given their initial beliefs and consumption history up until time $t$. Note that this expression is known to be the certainty equivalent for CARA preferences in a Gaussian environment \cite{mas1995microeconomic}.

\xhdr{User Learning}
When a user consumes an item $n$ she learns the realized value for that item. In our model, we incorporate the idea that learning the value of item $n$ gives a user information about similar items. 
%This is drawn from recent empirical evidence in \cite{schulz2019structured} that consumers learn how to navigate large choice sets using similarity-based generalization. 
We assume that learning about the value of item $n$ reveals more about the value associated to items that are closer to it, which captures the idea that trying an item provides more information about similar items than about dissimilar ones.
\par
In order to have a well-defined notion of similarity we need to define a distance function between items, which we define as $d(n,m):=\min\{ \lvert m - n \rvert ,N - \lvert m - n \rvert \}$ where $m$ and $n$ are indices of items in $\mathcal{J}$. This distance function is not intended to model the intricacies of a realistic product space, but instead to provide a stylized product space to help us understand the effects of informational spillovers on user behavior. We consider that the entry of $n$-th row and the ($m$)-th column of $\Sigma_i$ is given by $\sigma_i^2 \rho^{d(n,m)}$, and that of $\Sigma$ is given by $\sigma^2 \rho^{d(n,m)}$. The scalar $\rho \in [0,1]$ therefore impacts the covariance structure: a higher $\rho$ implies that learning the utility of $n$ is more informative about products nearby. Informativeness, for any $\rho \in (0,1)$, is decreasing in distance. The particular distance function that we utilize leads to this covariance structure being simple, where the $(n,n+1)$-th entry in the covariance matrix is $\sigma^{2} \rho$, the $(n,n+2)$-th entry is $\sigma^{2} \rho^2$, etc. \footnote{This exponential decay correlation structure can be related to the tenet of case-based similarity of \cite{gilboa1995case} -- see \cite{billot2008axiomatization} for an axiomatization of exponential similarity.}
\par
The precise updating rule is as follows. Recall that at time $t$ the user's consumption history is given by $C_{i}^{t}$ and we denote the utility realizations of these items as $c_t$. We denote $\mu_t$ as the initial mean beliefs the user has over the items in $C_{i}^{t}$ and $\mu_{N-t}$ as the initial mean beliefs the user has over the remaining $N-t$ items, $\mathcal{J} \setminus C_{i}^{t}$. We partition the covariance matrix as follows:
\[ \Sigma =  \left( \begin{array}{cc}
\Sigma_{(N-t, N-t)} & \Sigma_{(N-t,t)} \\
\Sigma_{(t,N-t)} & \Sigma_{(t,t)}
\end{array} \right)
\]
Thus, after consuming the items in $C_{i}^{t}$ the resulting beliefs over the remaining items are given by $\mathcal{N}(\bar{\mu}, \bar{\Sigma})$ where $\bar{\mu}$ and $\bar{\Sigma}$ are as follows:
\begin{align*}
\bar{\mu} \mid c_t = \mu_{N-t} + \Sigma_{(N-t,t)} \Sigma_{(t,t)}^{-1}(c_t - \mu_t) \\
\bar{\Sigma} \mid c_t = \Sigma_{(N-t,N-t)} - \Sigma_{(N-t,t)} \Sigma_{(t,t)}^{-1} \Sigma_{(t,N-t)}
\end{align*}
\begin{figure}[t]
\includegraphics[width=.3\linewidth]{figures/Example-Bubbles.pdf}
\captionof{figure}{Illustrative Example}
\label{fig:illustrative_example}
\end{figure}

\xhdr{An Illustrative Example} We illustrate the main intuitions of our model with a simple example. Suppose that there are four items: 0, 1, 2, 3. The items are in different places of the product space, where 0 is close to 1 and 3 but more distant from 2, as shown in Figure \ref{fig:illustrative_example}. The initial beliefs are given as follows:
\[ \mu = \left (\begin{array}{c}
\mathbb{E}[x_0] \\
\mathbb{E}[x_1] \\
\mathbb{E}[x_2] \\
\mathbb{E}[x_3]
\end{array}  \right) = \left (\begin{array}{c}
0 \\
0\\
0 \\
0
\end{array}  \right), \Sigma =  \sigma^{2} \left( \begin{array}{cccc}
\rho^{0} & \rho^{1} & \rho^{2} & \rho^{0} \\
\rho^{1} & \rho^{0} & \rho^{1} & \rho^{2} \\
\rho^{2} & \rho^{1} & \rho^{0} & \rho^{1} \\
\rho^{1} & \rho^{2} & \rho^{1} & \rho^{0} \\
\end{array} \right)
\]
In period 1, every item is ex-ante identical since they have the same mean and variance and so suppose that the user breaks the tie arbitrarily and consumes item 0. The underlying correlation structure implies that upon observing that $x_0=y$ the user will update beliefs about the remaining 3 items according to the previously specified updating rule. For concreteness, we suppose that $\sigma = 1$ and $\rho = 0.5$, but the intuitions hold for any value of $\sigma$ and $\rho > 0$. First, we consider the case when the realization of $y > 0$ and, for concreteness, specifically when $y = 0.5$ though the general intuitions hold for any $y > 0$. The resulting beliefs after observing $y$ are then as follows:
\[ \bar{\mu} =   \left (\begin{array}{c}
\mathbb{E}[x_1] \vspace{0.15cm} \\
\mathbb{E}[x_2] \vspace{0.15cm} \\
\mathbb{E}[x_3]
\end{array}  \right) =\left (\begin{array}{c}
\rho y  \vspace{0.15cm} \\
\rho^{2} y  \vspace{0.15cm} \\
 \rho y \\
\end{array} \right) =
\left (\begin{array}{c}
\frac{1}{4} \vspace{0.15cm} \\
\frac{1}{8}  \vspace{0.15cm} \\
\frac{1}{4}
\end{array}  \right), \bar{\Sigma} =  \left( \begin{array}{ccc}
\frac{3}{4} & \frac{3}{8} & 0 \vspace{0.15cm} \\
\frac{3}{8} & \frac{15}{16} & \frac{3}{8} \vspace{0.15cm}  \\
0 &\frac{3}{8} & \frac{3}{4}  \\
\end{array} \right)
\]
Thus, upon learning $y$, the user updates beliefs about the remaining items. Note that $\mathbb{E}[x_1] = \mathbb{E}[x_3] > \mathbb{E}[x_2]$ since item 0's value is more informative about similar items' values, items 1 and 3, than items further way in the product space such as item 2. Moreover, $\bar{\Sigma}_{11} = \bar{\Sigma}_{33} < \bar{\Sigma}_{22}$ as the uncertainty about items 1 and 3 is further reduced compared to item 2. Thus, since $y > 0$, the user in the next period will consume items nearby to item 0 since, even though initially she believed that all items had the same mean, the spillover from consuming item 0 leads her to believe that items 1 and 3 have higher expected valuations. Since both the mean is higher for these items and the variance is lower, the user will consume items 1 and 3 regardless of her risk aversion level.
\par 
Now we consider the case when item 0 ends up having a negative valuation so that $y = -0.5 < 0$. This results in $\mathbb{E}[x_1] = \mathbb{E}[x_3] = -\frac{1}{4} <  -\frac{1}{8} = \mathbb{E}[x_2]$ with $\bar{\Sigma}$ remaining the same as when $y = 0.5$. In this case the risk-aversion levels of the user determine the choice in the next period. If the user is risk-neutral ($\gamma = 0$), then she will go across the product space to consume item $2$ in the next period since it has a higher expected value. However, if she is sufficiently risk-averse then she still may consume item $1$ or $3$ since her uncertainty about these items is lower than item $2$. In particular, this will happen when 
\begin{align*}
\delta(3) = \delta(1) = \rho y - \frac{1}{2} \gamma \bar{\Sigma}_{11} > \rho^{2} y - \frac{1}{2} \gamma \bar{\Sigma}_{22} = \delta(2)
\end{align*}
Given the aforementioned parametrization and $y = -0.5$, the user will consume item $1$ or $3$ when $\gamma > \frac{4}{3}$ and will consume item $2$ when $\gamma < \frac{4}{3}$. Thus if the user is risk averse enough, then she might be willing to trade-off ex-ante lower expected values for lower risk and stick to consuming nearby items just because these items have lower uncertainty. 
\par 
This example illustrates the main mechanisms in our model that can lead to excessive consumption of similar items. Once the user finds items in the product space with high valuations she will update her beliefs positively about items in this portion of the product space and continue consuming these items regardless of her level of risk aversion. However, this same updating leads to a reduction in uncertainty of these items and so, if she is sufficiently risk-averse, she still may continue consuming items in this portion of the product space, even if she has bad experiences with them, since they are perceived to be less risky. 
\par

\xhdr{Recommendation.}
Our model of recommendation is stylized in order to provide qualitative insights into how recommendation shapes behavior, instead of focusing on the details of how recommender systems are implemented in practice. We model recommendation as giving users information about the valuation of the products.
\par

We will consider three cases. The case of primary interest is \textit{recommendation} where the recommender observes values accrued and knows $V$ but does not know neither $V_i$.\footnote{We do not consider the acquisition of information for the recommender to know $V$ and suppose that she has sufficient data to learn $V$ with arbitrary precision at $t = 0$.} However, the recommender does know the users' beliefs $\bar V_i$. Thus, at any given period, the recommender provides a personalized recommendation that combines the knowledge of the common value component $V$ with the user beliefs $\bar V_i$. Knowing the user's beliefs about her valuation of each item become crucial in this case: just providing the user with information about $V$ may change the user's original ranking, but, without considering the user's beliefs, she will not necessarily follow the recommendation.\footnote{The notion of recommendation that we consider is idealized where the recommendation does the Bayesian updating for users. As users face a very difficult and costly Bayesian updating instance, a useful recommendation system provides recommendations taking into account this aspect and combines the history of past consumption and value realizations with the user's beliefs. 
An alternative that is equivalent would be the recommender simply providing $V$ without knowing $V_i$. In this case, the recommendation will be that the user $i$ chooses $r_{i,t} \in \arg \max_{n\in \mathcal J\backslash C^T_i} \mathbb E[u(x_n) \mid C_i^T]$
but we assume the recommender provides full information about $V$. For instance, the recommender could display the whole distribution of utilities reported by other users or even its average, which is a good proxy for the common value component. It is possible for the recommender to only recommend the best item given such information, that is, the item with highest common value component. However, in that case, recommending only a single item generates costly Bayesian updating to the user and provides little guidance as to what she should indeed pick as the common value might be of little importance when compared to the idiosyncratic component. Therefore, we assume that the recommender reports the whole $V$, which results in higher expected welfare in choices and is not costly to implement.
The results would be equivalent, but this would shift the burden of Bayesian updating to users.}
\par

We further consider two cases that serve primarily as benchmarks. The first is \textit{no recommendation}, where users get no additional information about values and make consumption choices based on their beliefs and consumption history. This gives us a benchmark as to how users in our model would behave \textit{without} recommendation so that we can analyze what changes with the introduction of recommendation. The second is the \textit{oracle recommendation} where the recommender knows the true realized utility of each good for each user and can therefore recommend the best remaining good in every period. This gives us a full information benchmark, which is the optimal consumption path for a user if all uncertainty about their preferences was resolved. Comparison to the oracle regime benchmark provides an analog to the standard regret measures utilized in the multi-armed bandit literature, which look at the difference in the expected value of the ex-post optimal action and the expected value of actions that were taken.
\par

\xhdr{Simulation Details}. 
We analyze our model using numerical simulation since the sequential decision-making component of our model paired with the rich covariance structure between the items make it difficult to characterize optimal user behavior analytically. The Gaussian assumption allows for closed form belief updating which enables us to simulate our model but does not provide much help in analytical characterizations. 
\par

We explore how consumption patterns differ as we consider different recommendation regimes and report representative results from our simulations. We simulate our model over 100 populations of users with 100 users per population for each combination of parameters. 
%For every population and every statistic we calculate, we average over the users in the population to get a representative statistic for this population. 
A given set of parameters and a user are a single data point in our dataset.
\par

We simulate over risk-aversion parameters $\gamma \in \{ 0, 0.3, 0.6, 1, 5 \}$, \ standard-deviation $\sigma \in \{ 0.25, 0.5, 1.0, 2.0, 4.0 \}$, correlation parameters $\rho\in \{ 0, 0.1, 0.3, 0.5, 0.7, 0.9 \} $ and degree of idiosyncrasy of individual valuations $\beta \in \{ 0, 0.4, 0.8, 1, 2, 5\}$. The range of considered parameter values cover the relevant portions of the parameter space in order to provide full insight into the behavior of the model. 
\par
When we consider results varying a single parameter, we group the results over the other parameters and provide reports varying only the parameter of interest. We report results for a relatively small consumption history $T=20$ with a product space size $N=200$. Unless otherwise reported, our results are robust to increasing $N$.
\par

\begin{figure}
\begin{subfigure}{.5\linewidth}
  \centering
  \includegraphics[width=.9\linewidth]{figures/rho_pos_consumption_dist_N_200T_20_overall.pdf}
  \label{fig:sfig1}
\end{subfigure}%
\begin{subfigure}{.5\linewidth}
  \centering
  \includegraphics[width=.9\linewidth]{figures/rho_zero_consumption_dist_N_200T_20_overall.pdf}
  \label{fig:sfig2}
\end{subfigure}
\caption{Consumption path with (Left) and without (Right) correlation between values}
\label{fig:correlation_consumption_path}
\end{figure}

\section{Results}
\subsection{Local Consumption and Filter Bubbles}
We characterize ``filter bubble'' effects in our model as the degree to which users engage in \textit{local consumption}. We define local consumption in terms of the average consumption distance between the items consumed by the users at time $t-1$ and $t$. Thus, in the context of our model, filter bubble effects arise when the average consumption distance decreases over time and, across regimes, when the overall levels are lower for a given recommendation regime compared to another.


\begin{finding}\label{finding_local_consumption}
The impact of recommendation on local consumption:
\begin{enumerate}
\item When $\rho = 0$, there is no difference in consumption distance between the three recommendation regimes.
\item When $\rho > 0$, no recommendation induces more local consumption than both recommendation and oracle regimes. This effect is amplified as $\rho$ increases as well as when users are more risk averse ($\gamma$ increases)
\end{enumerate}
\end{finding}

First, the right panel of Figure \ref{fig:correlation_consumption_path} shows that, when $\rho = 0$, there is no difference in consumption distance between the three regimes. This is due to the fact that when $\rho = 0$, there is no reason that items that are close in the product space should have similar values and so the optimal consumption path does not depend on the similarity of the items. However this also means that users do not learn anything about neighboring products and so there is limited path-dependence in consumption. Not only is there no difference in the levels between the three regimes, but they all have the same, flat, average consumption distance path. This underscores the fact that such concerns about filter bubbles \textit{only} arise in a world where there is correlation between the realized utility of items. If there were no correlation between the realized utilities then there would be no reason for users to consume similar items and thus no narrowing effect, regardless of the information on the true utility of the items that users had.
\par
The top panel of Figure \ref{fig:correlation_consumption_path} shows that, when $\rho > 0$, both recommendation and no recommendation lead to increasingly local consumption compared to the oracle benchmark case. Further, the average consumption path between periods is \textit{decreasing} for the no recommendation case whereas it is \textit{increasing} for the oracle case. The recommendation regime decreases the degree of local consumption, but not as much as the oracle benchmark. Due to the correlation of values, the oracle consumption path exploits this and leads to the consumption of more similar products than in the case when $\rho = 0$. However, since these spillovers also impact user learning in the no recommendation case, users \textit{over-exploit} this and increasingly consume products similar to good products that they have consumed before. This is further illustrated by Figure \ref{fig:local_consumption_across_rho} which shows how the consumption paths in the oracle and no recommendation regimes vary as $\rho$ increases.
\par

Finally, this effect is further amplified as the level of risk aversion, $\gamma$, increases. Figure \ref{fig:no_rec_risk_aversion} shows how drastically the degree of local consumption increases as $\gamma$ increases. This is due to the fact that the spillovers not only affect the mean expected belief about quality but also the degree of uncertainty. Local consumption therefore leads to users having less uncertainty about certain areas of the product space and risk aversion may lead them to increasingly consume nearby products.

In sum, filter bubble effects can only arise when there is an inherent correlation between the realized utilities of the items in the product space. However, this is often the case in most environments where recommender systems are deployed. When there is a correlation between the realized utilities then filter bubbles can naturally arise due to the nature of how individuals acquire additional information about the remaining items. We have shown that, unless users are provided with additional information to guide their consumption choices, then these information spillovers and user risk-aversion can lead users into filter bubbles. When users consume high valued items, they exploit the underlying correlation across different products' values, stronger for similar products, which leads them to increasingly consume products in narrower and narrower portions of the product space. Risk aversion may lead users into performing local consumption even when they have a low valuation of nearby items just because they know what to expect from the item. Recommendation leads to these effects being mitigated by providing users with additional information on products outside the already explored portions of the product space. Further, if all uncertainty was resolved as in the oracle regime, then such behavior is not present.

\subsection{User Welfare and Product Diversity}
In this section we primarily focus on the impact of recommendation on user welfare and the overall diversity of the items that they consume.
\par 
While in the previous section we looked at the distance between consecutive items, in this section we focus on a diversity measure that considers the entire consumed set of items. The diversity measure we utilize is common in recommender system literature (e.g. \cite{ziegler2005improving}) which is the average normalized pairwise distance between the consumed products:
$$D_i:=\frac{1}{N}\frac{1}{T(T-1)}\sum_{n,m \in C_i^T: n \ne m} d(n,m)$$

\begin{finding}\label{finding_diversity}
The impact of recommendation on product diversity:
\begin{enumerate}
\item When $\rho = 0$, product diversity is the same across all three recommendation regimes;
\item When $\rho > 0$, product diversity decreases across all recommendation regimes but decreases the most in the no recommendation regime. This effect is amplified as $\rho$ increases as well as when users become more risk-averse.
\end{enumerate}
\end{finding}
\par 
Finding~\ref{finding_diversity} summarizes the main results on product diversity. As before, when there is no correlation between valuations, product diversity is the same across different recommendation regimes. The over-exploitation of information spillovers when $\rho > 0$ leads to product diversity being lowest in the no recommendation regime. As a result, this effect gets amplified as $\rho$ increases, which leads to the gap in diversity between the regimes to increase as $\rho$ increases. Figure \ref{fig:diversity_welfare_correlation} shows how diversity varies as $\rho$ increases across the three regimes that we consider. There is a similar increasing diversity gap as $\gamma$, or the level of risk-aversion, increases as can be seen in Figure \ref{fig:risk_aversion_diversity}. The mechanisms behind these effects have direct parallels to those discussed in the previous section since low average sequential consumption distance directly leads to low diversity.
\par 
We now study how recommendation impacts user welfare in our model. In our model users make consumption decisions that maximize their current period ex-ante utility that depends on their beliefs in that period. Thus, from an ex-ante perspective, they make optimal decisions, but our primary interest is in understanding how the ex-post, or realized, utility varies across regimes and parameter values. We define user's \textit{ex-post} welfare as the average of the realized values, controlling for the effect of $T$:
$$W_i:= \frac{1}{T}\sum_{n \in C_i^T} x_{i,n}$$

\begin{finding}\label{finding_welfare_gap}
The impact of recommendation on consumer welfare is as follows:
\begin{enumerate}
\item Under oracle recommendation, welfare is constant as $\rho$ increases.
\item Under no recommendation, welfare is increasing as $\rho$ increases.
\item Recommendation introduces welfare gains relative to no recommendation, but these gains are decreasing as $\rho$ increases. 
\end{enumerate}
\end{finding}
\par 
Finding~\ref{finding_welfare_gap} states our main findings of the impact of recommendation on ex-post welfare, which are drawn from Figure~\ref{fig:diversity_welfare_correlation}. The most interesting observation is that the value of recommendation decreases as $\rho$ decreases. The intuition for this is clear as recommendation provides users with information that allows them to better guide their decisions and increase welfare. However, as $\rho$ increases users get increasingly more information simply from consuming items since the realized utility is now more informative about the utility of nearby items. Thus, since consumption decisions themselves reveal information, the information that recommendation reveals provides less value.
\par 
Figure~\ref{fig:diversity_welfare_common_value} shows how diversity and welfare vary across recommendation regimes as the the strength of the common-value component, $\beta$, varies. As expected, when $\beta = 0$, these values are the same. However, as $\beta$ grows the idiosyncratic component of user valuations becomes increasingly less important and the recommendation regime begins to converge to the oracle regime. Indeed, we see that both product diversity and welfare increase under recommendation, whereas under no recommendation we see that there is a decrease in diversity with little changes to consumer welfare. This is intuitive as $\beta$ controls the informativeness of recommendation about user preferences.
\par 
One striking observation is that the decrease in diversity does not appear to be associated with a decline in welfare. Indeed, it appears that the opposite is the case - that low diversity is associated with higher welfare and vice versa. Thus, we next explore the relationship between welfare and diversity.
\begin{figure}
\begin{subfigure}{.45\linewidth}
  \includegraphics[width=.8\linewidth]{figures/rho_diversity_N_200_T_20.pdf}
\end{subfigure}
\begin{subfigure}{.45\linewidth}
  \includegraphics[width=.8\linewidth]{figures/rho_welfare_N_200_T_20.pdf}
\end{subfigure}
\caption{User Welfare and Diversity as the strength of correlation, $\rho$, varies}\label{fig:diversity_welfare_correlation}
\end{figure}


\begin{finding}\label{finding_diversity_welfare_corr}
In the no recommendation regime, diversity and welfare are:
\begin{enumerate}
\item Negatively correlated when users have no risk-aversion;
\item Uncorrelated when users have high levels of risk-aversion.
\end{enumerate}
In the recommendation regime, diversity and welfare are:
\begin{enumerate}
\item Uncorrelated when users have no risk-aversion;
\item Positively correlated when users have high levels of risk-aversion.
\end{enumerate}
In the oracle regime, diversity and welfare are always uncorrelated.
\end{finding}

Finding~\ref{finding_diversity_welfare_corr} summarizes our findings on the relationship between diversity and welfare. Figure \ref{fig:diversity_welfare_ra} shows how diversity and welfare correlate for the no recommendation case as we vary the degree of risk aversion. When there is no risk-aversion then there is a negative correlation between welfare and diversity. This is since, with no risk-aversion, in a given period a user will select the good that she currently believes has the highest expected value. High product diversity in this case can arise from a user who consumes an item she disliked and updates her beliefs about nearby items negatively. As a result, in the following period she will pick an item far away in the product space from the item that was previously consumed. If instead the user valued highly the item that she had consumed, then she is more likely to pick a nearby item. The information spillovers therefore lead to high product diversity being negatively correlated with welfare.
\par
This only happens since $\gamma = 0$ leads to users only caring about the expected value of the product. However, as we saw in Findings \ref{finding_local_consumption} and \ref{finding_diversity}, increasing $\gamma$ can lead to lower diversity and increasingly local consumption due to the fact that the degree of uncertainty now impacts users' choices. This weakens the negative relationship between diversity and welfare since both negative and positive experiences with a good reduce uncertainty about surrounding items. This leads to the inverted-U shape found in Figure \ref{fig:diversity_welfare_ra} when $\gamma$ is relatively large (e.g $\gamma = 5$) though diversity and welfare are virtually uncorrelated in the data. In the recommendation and oracle regimes, under risk neutrality ($\gamma=0$), welfare and diversity and uncorrelated, while under risk aversion ($\gamma=5$), it is possible to observe in Figure \ref{fig:diversity_welfare_ra_partial} an actual positive relation between diversity and welfare as recommendations are able to reduce uncertainty and facilitate exploration of the product space.
\begin{figure}[t]
\begin{subfigure}{.45\textwidth}
\includegraphics[width=1.05\linewidth]{figures/diversity_welfare_rn.pdf}
\end{subfigure}
\begin{subfigure}{.45\textwidth}
\includegraphics[width=1.05\linewidth]{figures/diversity_welfare_ra.pdf}
\end{subfigure}\\
\caption{Diversity vs. Welfare, \textbf{No-Recommendation}, $\gamma = 0$ (Left), Diversity vs. Welfare, $\gamma = 5$ (Right)}\label{fig:diversity_welfare_ra}
\end{figure}

\begin{figure}[t]
\begin{subfigure}{.45\textwidth}
\includegraphics[width=1.05\linewidth]{figures/diversity_welfare_rn_partial.pdf}
\end{subfigure}
\begin{subfigure}{.45\textwidth}
\includegraphics[width=1.05\linewidth]{figures/diversity_welfare_ra_partial.pdf}
\end{subfigure}\\
\caption{Diversity vs. Welfare, \textbf{Recommendation}, $\gamma = 0$ (Left), Diversity vs. Welfare, $\gamma = 5$ (Right)}\label{fig:diversity_welfare_ra_partial}
\end{figure}

\subsection{User Homogenization}
In this section, we focus on comparisons across users and investigate how the consumed set of items across users varies across different recommendation regimes and parameter values. In particular we look at the degree of \textit{homogenization} between users. Similar to other papers that study the degree of homogenization in recommender systems (e.g. \cite{chaney2018algorithmic}) we measure homogeneity via the Jaccard index between the consumption sets of users:
\begin{align*}
H:=\frac{1}{|I|(|I|-1)}\sum_{i,j \in I: i \ne j}d_J(C_i^T,C_j^T)
\end{align*}
where $d_J$ denotes the Jaccard index and $H \in [0,1]$.
\begin{finding}\label{finding_homogeneity}
The impact of recommendation on homogeneity is as follows:
\begin{enumerate}
\item Highest under recommendation and lowest under no recommendation;
\item Increasing in $\beta$, or the weight of the common-value component;
\item Decreasing in $\rho$ for partial recommendation, but weakly increasing in $\rho$ for no recommendation.
\end{enumerate}
\end{finding}
Finding~\ref{finding_homogeneity} summarizes our findings on the impact of recommendation on user homogeneity. First, we study how the degree of homogenization varies as we increase $\beta$, the weight of the common value component. As $\beta$ increases we expect that users should become increasingly homogeneous as the realized utilities of the items are now increasingly similar. Figure \ref{fig:beta_homo} confirms that as the weight of the common-value component, $\beta$, increases users consume increasingly similar sets of items. The homogenization effect is strongest under the recommendation regime since the revelation of the common-value component induces users to consume products in similar areas of the product space. As $\beta$ increases, some amount of homogenization is optimal as can be seen from the oracle case. However, since users in the no recommendation regime do not know the common-value component they engage in local consumption in different areas of the product space which leads to less than optimal homogeneity.
\par
We next study how the degree of homogeneity varies as $\rho$ increases. Figure \ref{fig:cor_homo} shows how homogeneity varies as $\rho$ increases across the different recommendation regimes. The most striking finding is that homogeneity decreases as $\rho$ increases in the recommendation regime. As was highlighted in Findings \ref{finding_local_consumption} and \ref{finding_diversity}, the degree of local consumption increases with $\rho$. Even though the revelation of the common-value component induces them to search in similar parts of the product space, their idiosyncratic components induce them to consume products in a more localized area of the product space as $\rho$ increases which leads to a decline in homogeneity.

\section{Recommender System Evaluation}
In this section we discuss how the insights from our model of user decision-making can be used to inform the evaluation and design of recommender systems.
\par

The classic approach to recommender system evaluation is to predict user ratings for items and compare how accurate this prediction is to recorded ratings data, either explicitly given by users or inferred from behavioral data. Further, given this predicted rating, the recommender system should recommend to users the items with the highest predicted rating \cite{adomavicius2005toward}.
\par


There has been a recent movement away from such evaluation measures due to the observation that accurate recommendations are not necessarily useful recommendations \cite{mcnee2006being}. Our model illustrates why this could be the case. Consider the domain of movie recommendation and suppose a user has just watched the movie \textit{John Wick} and rated it highly. A recommender system attempting to predict user ratings may then predict that this user is very likely to enjoy the sequel, \textit{John Wick: Chapter Two}, as well. However, the user herself may also have made this inference since the two movies are very similar to each other and, after enjoying \textit{John Wick}, the user will update her beliefs about \textit{John Wick: Chapter Two}. Thus, if the recommender system recommends this movie to the user then this recommendation is not useful since the recommendation gives the user little information that they did not already know. The key insight is that the recommendation is not useful since \textit{it ignores the inference the user themselves made and their updated beliefs}. The user may watch \textit{John Wick: Chapter Two}, then, even without recommendation and the value of the recommendation was small.
\par

As a result, several alternative recommendation evaluation metrics, such as serendipity \cite{kotkov2016survey}, coverage \cite{ge2010beyond}, novelty \cite{vargas2011rank}, and many others have been proposed in order to develop recommender systems that produce recommendations that are more useful for users. We use our model to motivate an alternative recommendation evaluation metric that provides an alternative perspective on the serendipity metric proposed in the literature. Serendipitous recommendations are said to ``have the quality of being both unexpected and useful" \cite{maksai2015predicting} and we use this as a guide towards our definition.

Our approach to serendipity comes from the intuition that a recommendation is only useful insofar as it leads a user to take a better action than she would without recommendation. This approach stems from the idea that the value of recommendation is the marginal utility gain that a user receives from recommendation.\footnote{This notion is similar to the ``value of information" notion utilized in the literature in economics on the pricing of information goods \cite{bergemann2018design}.} This intuitively should come from providing information about products that, without additional information, users assign a low likelihood to being the best products currently available for them to consume and so would not consume them without these being recommended.
\par

This differs from other approaches in the literature since it depends on the beliefs of the users and what item the user would consume without recommendation. For instance, \cite{vargas2011rank, kaminskas2014measuring} propose unexpectedness metrics that look at the dissimilarity of the recommended items to what the recommender already knows the user likes either via a content-based approach or a collaborative-based approach. This depends only on the resulting item-set that the users get recommended and not necessarily user beliefs or the change in user action. \cite{adamopoulos2014unexpectedness} use an expected utility approach but use this to measure unexpectedness in terms of deviations from the ``expected consideration set" of a user and do not explicitly incorporate user beliefs.
\par

Our model allows us to give precise definitions for what it means for a recommendation to be \textit{unexpected} and \textit{useful}. Unexpected recommendations are given by recommended items that, given their beliefs, users have low expected utility from consuming, irregardless of what the true realized value is. Recall from Findings \ref{finding_local_consumption} and \ref{finding_diversity_welfare_corr} that, particularly if users are risk-averse, they will consume content in increasingly narrow portions of the product space and this may be detrimental to their realized welfare. As a result, identifying the portions of the product space where users may have significant uncertainty and utilizing recommendation to reduce this uncertainty can be welfare-enhancing for users. This is especially the case if this reduction of uncertainty leads users to explore under-explored portions of the product space where true realized utility is high, but expected utility, given beliefs, is low before recommendation.
\par

Useful recommendations are those that give users information that would change the item they consume relative to no recommendation and lead to an increase in expected utility given the updated beliefs when comparing the planned choice before and after the recommendation. In the previous example, recommending \textit{John Wick: Chapter Two} is not useful precisely because the expected utility before and after recommendation is the same since it does not change behavior. The most useful recommendations are those that (1) are informative given user beliefs and (2) induce users to consume unexpected goods.
\par

Operationalizing this approach requires the collection of data that is not traditionally collected for recommender systems. In particular, it becomes important to understand what choices users would make \textit{without recommendation} and to collect data on user beliefs, which can partially be inferred by observed choices. \cite{jiang2014choice, saavedra2016choice} argue that choice-based approaches alone can help in designing better RS, though they argue for these approaches because they do a better job at providing more accurate recommendations. Instead, we stress that choice-based approaches are useful not as recommendations but as a baseline for what recommendations would be useful.\footnote{Additionally, user beliefs and user choices contain information that may not have been observed by the recommender but is observed by the user. Ratings, reviews, friends, there are many sources of information affecting a person's beliefs and choices that are unobservable by recommender systems. Other consumption choices, e.g. movies seen in the cinema, change beliefs about for instance how good a director is, which affect beliefs about other movies and guide choices on a streaming platform that is unable to observe these data. However, this information can be inferred by the platform by collecting both beliefs and consumption choices.}
\par

Furthermore, our model shows the importance of understanding the underlying user preferences and the nature of the product space in the setting in which a recommender system is deployed since that will change the consumption patterns of users. For instance, understanding the degree of correlation of utilities and the level of risk-aversion become crucial in understanding how users will make consumption choices without recommendation and what kind of recommendation would be useful for users.
\par



\section{Conclusion}
We have studied a model of user decision-making in the context of markets where recommender systems are traditionally deployed in order to understand the consequences of recommender systems on consumption patterns of users. The key components of the model are that users make decisions under uncertainty and that consuming a good leads to informational spillovers about similar items. We have shown that this alone can generate consumption behavior that is consistent with filter bubble effects that have traditionally been associated with personalized recommender systems. This effect is weakened once we allow for recommendation, but not as much as if users have full information. We have further explored the relationship between consumption diversity and user welfare and found that they are negatively correlated, implying that striving for consumption diversity is not necessarily welfare-improving for users. Finally, we have shown that recommendation can lead to increases in user homogenization both relative to no recommendation and full information since it concentrates users' consumption in portions of the product space with the highest utility according to the limits of what the recommender can predict about the utility of the good.
\par 
We leave for future work exploring to what extent user behavior changes if we drop the myopia assumption, so that users themselves are engaging in explicit exploration. This likely interacts with recommendation in interesting ways as the recommendations may provide information that can complement or crowd-out user-led exploration. Even under the myopia assumption, understanding which recommendation system is optimal is an interesting and challenging problem. Further, our results point to an important consideration in the literature on incentivizing exploration which is that the exploration faced by the recommender itself likely interacts in interesting and important ways with the choice problem faced by the users.
\par 
Overall, understanding how users make choices in online platforms and how recommender systems impact these choices is important to understand. Understanding these issues allows us to better understand the broader social and economic impact of such systems as well as can provide a guide to improve the design of such systems.
% Bibliography
\bibliographystyle{ACM-Reference-Format}
\bibliography{refs}

\end{document}
